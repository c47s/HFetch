\documentclass[twoside,a4paper,12pt]{report}
\usepackage[utf8]{inputenc}

\usepackage[style=numeric,backend=biber]{biblatex}
\bibliography{bibliography}

\usepackage{listings}
\lstset{language=bash}

\usepackage{appendix}

\title{HFetch: A User-Friendly API Client for the Schoology LMS}
\author{Nathaniel Waterman}
\date{March 3, 2018}

\begin{document}

\maketitle

\begin{abstract}
Schoology is popular and powerful, but it takes time to check for homework posted there.
This was the reason for designing HFetch,
	which uses a RESTful API to fetch, display, and provide details about
	assignments from Schoology in 2 clicks, or 1 if it is launched automatically.

The introduction to Schoology's API documentation mentions that there were no Schoology API clients at the time it was written, but an implementation in Java was found on GitHub (rvanasa/schoology-api).
This paper intends to inform the reader about HFetch, seemingly the second such API client.
\end{abstract}

\tableofcontents

\chapter{Introduction}
Schoology is a learning management system (LMS), a website on which teachers can post assignments and resources for their students to access anywhere.\cite{scgy-promo}
Navigating to Schoology is inconvenient,
	and once on the site,
	all that is usually needed is the sidebar,
	which lists upcoming assignments.
It can be hard to remember to check Schoology,
	so an automated utility is extremely useful.
Because of this,
	a Unix shell script\cite{getassgns} was written which fetches assignments from the website,
	and an AppleScript App which displays the assignment information in a user-friendly GUI.
Bash and AppleScript, a MacOS scripting language, were chosen for this task because they integrate well and both run on MacOS,
	as the school that HFetch is intended for has an abundance of Macs.

Schoology has a RESTful API in order to allow for the creation of tools to enrich the Schoology experience.\cite{api-docs}
REST is a standard for web services which allows for the access and manipulation of web resources
	by way of a set of stateless,
	(each message is independent of the others) predefined operations.\cite{rest}

\chapter{Problem}
HFetch was designed to solve the problem of students forgetting to check Schoology for their assignments. This ruins one of the main benefits of a learning management system: Teachers are able to post homework without explicitly telling the class, and can expect each student to complete the assignment. Failure to complete homework, in this case due to forgetfulness, can sabotage learning and ruin grades.

\chapter{Solution}
HFetch is the solution to the aforementioned problem of forgotten homework.
It runs a shell script which downloads course information,
	then uses this to determine the IDs of the courses
	and goes on to query the server as to the assignments in each of these courses.
Once the assignments have been obtained,
	the script passes them on to HFetch to be displayed and exits.
If, however jq\cite{jq}, a JSON parser, isn't installed or the files storing the user's API credentials are missing, 
	the shell script returns an error which HFetch identifies and leads the user through the installation of jq
	and possibly the package manager Homebrew,
	or the obtaining and entry of their credentials,
	passing these to the shell script,
	which saves them on the disk.
\section{System Model}
\subsection{AppleScript App}
The AppleScript frontend requires MacOS to run.

\subsection{Shell Script}
The Bash Script must have a Unix-like environment with bash and jq installed, or an internet connection and package manager with which to download these dependencies.

\chapter{Technical Challenges}
A few lessons were learned while designing HFetch:

When testing the shell script, it was discovered that requests to the server must be sent with increments of at least 1 second, as the PHP time() function returns the time in seconds, not milliseconds.
Also, OAuth was found to be uncompromising in security - when the API credentials on disk became corrupted (somehow), jq\cite{jq} failed to parse the server's responses due to them being error messages instead of JSON. It took much troubleshooting to find that the root of the problem was not jq at all, but a credential issue.
The shell script uses jq to parse the JSON from the API, but since jq is not installed by default on MacOS, it had to be installed during setup.
However, as MacOS does not have a package manager, the package manager Homebrew had to be installed in order to install jq!

\chapter{Conclusions}
These programs make homework easier to remember,
	and allow for automation, such as with the cron utility\cite{cron} or Automator\cite{automator} on MacOS.
With cron, for instance, the app's executable could be scheduled to run at certain times and remind the student of their upcoming assignments.
Furthermore, with Automator, HFetch could be run with a keystroke or as a dictation command (by speaking).

In the future, a user interface should be created for Linux as the current one is written in AppleScript,
	new versions of this program should be created for Windows and iOS,
	and a similar client for the Powerschool API should be created or added as a part of HFetch.

\printbibliography

\appendix
\appendixpage
\addappheadtotoc
\chapter{Shell Script (getassgns)}
\input{chapters/getassgns}

\end{document}