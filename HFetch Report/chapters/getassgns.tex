\begin{lstlisting}
#!/bin/bash

interactive=1
justDoCredentials=0
tty=$(tty) # tty to direct interactive output to

while getopts "gret:" opt; do
	case "$opt" in
		g) # GUI mode - non-interactive
		interactive=0
		tty="/dev/null"
		;;
		r) # Delete the API credential files
		rm "$HOME/.scgyKey"
		rm "$HOME/.scgySecret"
		;;
		e) # Exit after modifying credentials
		justDoCredentials=1
		;;
		t) # Specify tty for interactive output
		tty="$OPTARG"
		;;
		*) # Invalid option
		echo "usage: $0 [-gre] [-t tty]"
		;;
	esac
done


# Make sure we can find jq if it is installed
PATH=$PATH":/usr/local/bin/"
jq --help > /dev/null 2> /dev/null || { # Test if jq is installed
	echo "Coulndn't find jq." > /dev/stderr
	if [ $interactive -eq 1 ]; then
		echo "Install it?"
		echo -n "(yes or no?) "
		read -n 1 answer
		read

		if [ "$answer" != "y" ]; then
			exit 5 # 5 - User decided not to install jq
		fi
		# Try every possible way to install jq
		brew install jq || \
		pkg install jq || \
		sudo bash -c '
			apt-get install jq ||
			dnf install jq ||
			zypper install jq ||
			pacman -Sy jq'
	else
		exit 4 # 4 - The GUI needs to install jq
	fi
}



# Try to read the credentials. If they are missing, prompt the user.
# If the user doesn't respond (or we're in GUI mode),
# exit with an error - 2 if failed while trying to read Consumer Key,
# 3 if failed while trying to read Consumer Secret.

read consumerKey < "$HOME/.scgyKey" || {
	echo -n "Consumer Key: " > "$tty"
	read -t 5 && echo "$REPLY" > "$HOME/.scgyKey" || exit 2
	# 2 - Missing Consumer Key
}

read consumerSecret < "$HOME/.scgySecret" || {
	echo -n "Consumer Secret: " > "$tty"
	read -t 5 && echo "$REPLY" > "$HOME/.scgySecret" || exit 2
	# 2 - Missing Consumer Secret
}

[ $justDoCredentials -eq 1 ] && exit
# Stop if option -e (exit after modifying credentials) was specified



IDs=( $(
curl https://api.schoology.com/v1/users/25493325/sections -sH "$(php -r "echo 'authorization: OAuth realm=\"Schoology API\", oauth_consumer_key=\"$consumerKey\", oauth_token=\"\", oauth_nonce=\"'; echo uniqid(); echo '\", oauth_timestamp=\"'; echo time(); echo '\", oauth_signature_method=\"PLAINTEXT\", oauth_version=\"1.0\", oauth_signature=\"$consumerSecret%26\"'; ")"\
 | jq -r .section[].id
) ) # Obtain the IDs of the user's classes

length=${#IDs[@]} # Find how many classes the user has

[ $interactive -eq 1 ] && tput sc > "$tty"
# If we're in interactive mode, save the current cursor position so
# we can return here in order to overwrite
# the previous progress percentage

result=$(

for i in "${!IDs[@]}"; do

	[ $interactive -eq 1 ] && tput rc > "$tty"
	echo -n $((i * 100 / length))"%" > "$tty"
	# Display progress percentage

	# Get each class' assignments, try again if this fails
	until curl https://api.schoology.com/v1/sections/"${IDs[$i]}"/assignments -sfH "$(php -r "echo 'authorization: OAuth realm=\"Schoology API\", oauth_consumer_key=\"$consumerKey\", oauth_token=\"\", oauth_nonce=\"'; echo uniqid(); echo '\", oauth_timestamp=\"'; echo time(); echo '\", oauth_signature_method=\"PLAINTEXT\", oauth_version=\"1.0\", oauth_signature=\"$consumerSecret%26\"'; ")"
	do
		sleep 0.1
	done

done

)




if [ $interactive -eq 1 ]; then

	tput rc > "$tty"

	{
	echo "Name\`Due Date"

	# Parse the json returned by the server for the due date and
	# name, keeping only ones that are due in the future
 	echo "$result" | jq -r '.assignment[] |
	select(
	(.due | strptime("%Y-%m-%d %H:%M:%S") | mktime)?
	 > 
	($date | strptime("%Y-%m-%d %H:%M:%S") | mktime)
	)
	|
	"\(.title):`\(.due)"' \
	--arg date "$(date +"%C%y-%m-$(($(date +%d))) %H:%M:%S")"
	} | column -ts "\`" > "$tty"

else

	# Parse the json returned by the server for the due date and
	# name, keeping only ones that are due in the future
	result=$(
	echo "$result" | jq -r '.assignment[] |
	select(
	(.due | strptime("%Y-%m-%d %H:%M:%S") | mktime)?
	 > 
	($date | strptime("%Y-%m-%d %H:%M:%S") | mktime)
	)
	|
	"Due \(.due):: \"\(.title)\": due \(.due), URL \(.web_url)"'\
	--arg date "$(date +"%C%y-%m-$(($(date +%d))) %H:%M:%S")" | \
	sort
	)



	# Sort the assignments by due date
	tempIFS="$IFS"
	IFS='
'

	for line in $result; do
		echo "${line#*:: *}"
	done

	IFS="$tempIFS"

fi
\end{lstlisting}